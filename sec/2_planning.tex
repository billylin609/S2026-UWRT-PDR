\section{Administrative Information}
\label{sec:plan}

\subsection{Team Resources}

The team operates from a dedicated design bay at the University of Waterloo, equipped with mechanical and prototyping. Additional support from university facilities, including the machine shop and paint room, enables complex manufacturing. Funding is sourced from university organizations such as WEEF and EngSoc, along with industry sponsors like Kenesto, QNX, and ProtoSpace Mfg, supporting prototyping, testing, and team operations. A financial statement is detailed in figure \ref{fig:uwrt_balance}. This year, our budget is allocated to three main areas: upgrading specific rover functionality such as wheels for improved grip on rocky terrain, acquiring higher-performance components like high-torque motors, and maintaining spare components for failures during testing.

 \subsection{Project Managment Plan}

Upon release of the URC 2026 requirements, our team break down our rover development cycle into three interconnected phases: functional validation, feature integration, and system-level testing. After PDR submission, all subsystems complete independent functional testing to validate core component performance. Prior to System Acceptance Review, we will develop a minimal viable product demonstrating core system capability across navigation, manipulation, and science tasks. After MVP validation, we will fix stability issues and conducts final system-level testing for competition readiness. The team's project schedule is detailed in the Gantt chart (Figure \ref{fig:uwrt_gantt}), which specifies responsible subteams, task dependencies, and critical dates.Confluence serves as the primary knowledge management system for technical documentation and meeting minutes. 

Integration follows a structured bottom-up approach where subsystems are independently validated before integrated, highlighting modularity in design. System validation occurs through three progressive stages: Software-in-the-Loop testing using Gazebo simulation for algorithm validation, Hardware-in-the-Loop testing with emulated sensors for system behavior performance, and System-Level testing at the Canadesys lunar facility to evaluate rover performance in competition-realistic environments. Testing schedules for each subsystem are labeled into figure\ref{fig:uwrt_gantt}.

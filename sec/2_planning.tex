\section{Administrative Information}
\label{sec:plan}

\subsection{Team Resources}

The team operates from a dedicated design bay at the University of Waterloo, equipped with mechanical and hardware prototyping tools. Additional support from university facilities, including the machine shop and paint room, enables complex manufacturing. Funding is sourced from university organizations such as WEEF and EngSoc, along with industry sponsors like Kenesto, QNX, and ProtoSpace Mfg, supporting prototyping, testing, and team operations. A financial statement is detailed in figure \ref{fig:uwrt_balance}. This year, our budget is allocated to three main areas: upgrading specific rover functionality such as wheel designs and including an actuated swivel DoFs for improved grip and controls on rocky terrain, exploring development of more bespoke systems like integrated swerve actuators, and maintaining spare components for failures during testing.

 \subsection{Project Managment Plan}

Upon URC 2026 requirements release, UWRT's rover development was organized into three interconnected phases: functional validation, feature integration, and system-level testing. All subsystems are tested for independent functionality before releasing the PDR; prior to SAR, a minimal viable product (MVP) demonstrating core navigation, manipulation, and science capabilities will be validated and recorded for the video submission; post-MVP, system performance and planned mission repeatability are addressed through final system-level testing for competition readiness. The project timeline, responsible subteams, task dependencies, and critical dates are detailed in the Gantt chart (Figure \ref{fig:uwrt_gantt}), with technical documentation and meeting minutes maintained in Confluence. Integration follows a structured bottom-up approach where subsystems are independently validated before full system integration. System validation occurs through three progressive stages: Software-in-the-Loop testing using Gazebo simulation for algorithm validation, Hardware-in-the-Loop testing with emulated sensors for system behavior assessment, and System-Level testing at the Canadensys lunar facility to evaluate performance in competition-realistic environments.
\section{Technical Design}
\label{sec:design}

\subsection{System Overview}
\label{subsec:sys_overview}

As illustrated in Figure \ref{fig:rover_sys_arch}, our rover is designed with three primary subsystems: a 35kg drivetrain with 6-wheel rocker bogie suspension powered by a 48V battery, a 6-DoF manipulator with brushless motor-encoder pairs, and a science payload featuring microscope imaging, environmental sensors, and soil sampling capabilities. Most of the rover hardware design is kept from the previous competition cycle, but the goal for this year is to make the design more like a product than a prototype.

\subsection{Season 2026 Updates}
\label{subsec:update}

\subsubsection{Compute Module}
This competition season, our team has welcomed QNX as a key sponsor in our rover development. QNX provides a high-safety, low-latency, real-time operating system running on Raspberry Pi 4B boards with comprehensive support packages. Our key architectural change this year is to replace all low-level STM microcontrollers with two RPi boards serving as I/O expansion modules, enabling preprocessing of sensor data before transmission to Jetson, our main compute module. This preprocessing layer improves system reliability and latency. We continue to use Jetson with ROS2 Humble as our main compute platform, as this solution has proven reliable and effective in previous seasons.

\subsubsection{Power System}

\subsubsection{Communication}

This year instead of implementing both LCM and DDS for comms although the hardware are the same, we are going to unify all the comms.

\subsubsection{Drivetrain}

\subsubsection{Arm}

\subsubsection{Science}

\subsubsection{Ground Station}
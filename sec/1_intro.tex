\section{Introduction}
\label{sec:intro}

The University of Waterloo Robotics Team (UWRT) is a dynamic, student-led team of over 30 members from engineering, math, and science programs, including Mechatronics Engineering, Systems Design Engineering, and Computer Science. A multifaceted group with 70\% of members joining to extend diverse backgrounds of highschool robotics enrichment in FIRST, VEX and other forms, UWRT members are inspired by real world applications of robotics with tangible lasting impacts. The team operates under a three-level hierarchy: the Team Lead coordinates with administrators and sponsors, the Safety Captain managed safety requirements and trainings, and the Business Lead take care of the outreach and relations with other organization. An Architecture Decision Committee of senior members approves all major architectural decisions, with general team members supporting across five functional subteams: mechanical, electrical, firmware, software, and business. The team structures can be visulized in figure \ref{fig:uwrt_team_structure}.

In the past, the team has participated in the Autonomous Robot Racing competition, mini sumo competition, Intelligent Ground Vehicle competition, and has participated in the University Rover Challenge as our main competition since 2010. UWRT Alumni go on to found industry leading robotics organizations, fostering  team history that ripples through and connects current members to exciting opportunities.

Within the team's two-level hierarchy, team Leads oversee high-level rover architecture, system integration, roadmapping, and external communications. Subteam Leads focus on subsystem implementation and onboarding, ensuring that member designs align with the team’s objectives. New members benefit from onboarding training, shadowing opportunities, and personalized mentorship. UWRT actively participates in Open House events and community outreach initiatives, inspiring future generations through programs and tours for elementary and middle school students, proudly representing the University of Waterloo and its mission. 

The team operates from a dedicated design bay at the University of Waterloo, equipped with mechanical and prototyping. Additional support from university facilities, including the machine shop and paint room, enables complex manufacturing. Funding is sourced from university organizations such as WEEF and EngSoc, along with industry sponsors like Kenesto, QNX, and ProtoSpace Mfg, supporting prototyping, testing, and team operations. A financial statement is detailed in Table 1.

